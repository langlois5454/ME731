\documentclass[french]{article}
\usepackage[utf8]{inputenc}
\usepackage{babel}
\usepackage{supertabular}
\usepackage[hyphens]{url}

\title{ME731\\Syllabus}
\author{David Langlois}
\date{Août 2021}

\begin{document}

\maketitle

\section{Objectif}

La série de modules 731, 831, 931 vise à revoir les notions informatiques abordées au Lycée, en SNT et NSI.

Considérant que ces notions ont déjà été abordées auparavant, on insistera en plus sur des éléments transversaux : production de documentation, jeux de test pertinents, etc.

De même, au-delà des notions revues, des exemples de difficultés, d'activités et d'exercices avec les élèves seront donnés.

\section{Modalités}

Le module sera organisé en séances de TD avec machine, combinant apports notionnels, échanges, activités de groupe, exercices sur feuille et machine.

\section{Durée}

Le module dure 114 heures sur des créneaux de 2h ou 3h.

\section{Intervenants}

les intervenants seront~:

\begin{itemize}
    \item Etienne Boyaval (EB)
    \item Sylvain Contassot (SC)
    \item David Langlois (DL)
    \item Yannick Parmentier (YP)
    \item Nicolas de Rugy Althere (NdRA)
\end{itemize}



\section{Documents institutionnels de référence}

\begin{itemize}
    \item Le programme NSI de Première~:\\ \url{https://cache.media.eduscol.education.fr/file/SP1-MEN-22-1-2019/26/8/spe633_annexe_1063268.pdf}
    \item Le programme NSI de Terminale~:\\ \url{https://cache.media.eduscol.education.fr/file/SPE8_MENJ_25_7_2019/93/3/spe247_annexe_1158933.pdf}
    \item Documents d'accompagnement~:\\ \url{https://eduscol.education.fr/2068/programmes-et-ressources-en-numerique-et-sciences-informatiques-voie-g}
\end{itemize}


\section{Contenu pédagogique}

Le module 731 aborde les points suivants~:

\begin{itemize}
\item Système (SYST, 8h)
\begin{itemize}
\item environnement de travail du master, de l'élève en lycée, système d'exploitation (dont Linux), gestion de processus, mémoire, machine virtuelle
\end{itemize}

\item Représentation numérique de l'information (REPINFO, 20h)
\begin{itemize}
\item Principe binaire ; représentation des valeurs numériques, naturelles, relatives, réelles ; représentation des caractères, des images, du son, des vidéos ; compression ; cryptographie ; théorie de l'information ; langages de description : XML, JSON ; sérialisation 
\end{itemize}
\item Architecture des ordinateurs (ARCHI, 32h) 
\begin{itemize}
\item Du transistor aux portes logiques ; algèbre booléenne ; circuits séquentiels ; circuits de mémoire ; machine de Turing ; assembleur ; éléments de compilation ; système ; robotique 
\end{itemize}
\item Modélisation (MODEL, 32h) 
\begin{itemize}
\item Type Abstrait de Données, liste, pile, file, implémentations ; tableaux ; tableaux associatifs ; tables de hachage ; dictionnaires ; arbres ; graphes ; bases de données \end{itemize}

\item Algorithmes \& Langages (ALGOPROG, 22h, domaine qui sera renforcé en semestre 8) 
\begin{itemize}
\item Les gammes des algorithmes (ce que tout informaticien devrait savoir) ; algorithmes sur les listes, les tableaux ; tris ;  
\item Programmation en python : programmation impérative, types de base
\end{itemize}

\end{itemize}

\section{Séances}

Voici la liste des séances, incluant le thème de la séance et un bref descriptif du contenu. L'ordonnancement et le contenu exact des séances sont donnés à titre indicatif et pourront évoluer en fonction des contraintes pédagogiques et organisationnelles.

\begin{center}

\tablefirsthead{%
\hline
Numéro de Séance & Thème & Contenu & Intervenant \\ \hline
}
\tablehead{%
\hline
\multicolumn{4}{|l|}{\small\sl suite de la page précédente}\\
\hline
Numéro de Séance & Thème & Contenu & Intervenant \\ \hline
}
\tabletail{%
\hline
\multicolumn{4}{|r|}{\small\sl voir page suivante}\\
\hline}
\tablelasttail{\hline}


\begin{supertabular}{|p{1.5cm}|c|p{6cm}|c|}
1 & SYST & mise en place environnement de travail & YP \\ \hline
2 & SYST & mise en place environnement de travail & YP \\ \hline
3 & REPINFO & introduction (langages, machine, algorithmes), pourquoi le binaire ? Notion de codage, les nombres entiers : codage en base 2, algorithmes passage 2 - 10 & EB \\ \hline
4 & REPINFO & les nombres relatifs : bit de signe, complément à 2, décalage & EB \\ \hline
5 & REPINFO & les flottants & DL \\ \hline
6 & REPINFO & les caractères, fichiers textes, binaires & DL \\ \hline
7 & REPINFO & théorie de l'information & SC \\ \hline
8 & REPINFO & les images & SC \\ \hline
9 & REPINFO & le son & SC \\ \hline
10 & REPINFO & TP son et image & SC \\ \hline
11 & REPINFO & cryptographie & SC \\ \hline
12 & REPINFO & cryptographie & SC \\ \hline
13 & ARCHI & rappel : ordinateur et binaire (intro archi), algèbre booléenne, calcul, simplification, table de karnaugh & DL \\ \hline
14 & ARCHI & le transistor et la porte non & DL \\ \hline
15 & ARCHI & porte ET OU & DL \\ \hline
16 & ARCHI & équivallence expression booléenne, conception de circuits & DL \\ \hline
17 & ARCHI & du circuit à la table de vérité & DL \\ \hline
18 & ARCHI & de la table de vérité au circuit & DL \\ \hline
19 & ARCHI & multiplexeur, somme arithmétique & DL \\ \hline
20 & ARCHI & mémoire & DL \\ \hline
21 & ARCHI & architecture von Neumann & NdRA \\ \hline
22 & ARCHI & architecture von Neumann & NdRA \\ \hline

23 & ARCHI & machine de Turing \& modèle de calcul & NdRA \\ \hline

24 & ARCHI & machine de Turing \& modèle de calcul & NdRA \\ \hline
25 & ARCHI & assembleur & EB \\ \hline
26 & ARCHI & assembleur & EB \\ \hline
27 & SYST & systeme, gestion de processus, mémoire. & YP \\ \hline
28 & SYST & systeme, gestion de processus, mémoire. & YP \\ \hline
29 & ARCHI & éléments de robotique & DB ? \\ \hline
30 & ARCHI & éléments de robotique & DB ? \\ \hline
31 & ALGOPROG & Focus python & YP \\ \hline
32 & ALGOPROG & Focus python & YP \\ \hline
33 & ALGOPROG & Focus python & YP \\ \hline
34 & MODEL & variable et type (retour sur codage) & YP \\ \hline
35 & MODEL & ensemble de données : tableau & YP \\ \hline
36 & MODEL & ensemble de données : tableau 2D & YP \\ \hline
37 & MODEL & python : tuple, liste (tableau, en fait) & YP \\ \hline
38 & MODEL & calcul de bit (codé 8 infos binaires sur un octet) & YP \\ \hline
39 & MODEL & modéliser les données d'un problème & EB \\ \hline
40 & MODEL & modéliser les données d'un problème & EB \\ \hline
41 & MODEL & modéliser les données d'un problème & EB \\ \hline
42 & MODEL & modéliser les données d'un problème & EB \\ \hline
43 & MODEL & structure de données, montrer qu'on peut utiliser des types composés, avec l'opérateur . & YP \\ \hline
44 & MODEL & listes chainées, pile, file, TAD : ne parler que de la structure, de l'intérêt pour la modélisation & YP \\ \hline
45 & MODEL & tables de hachage, dictionnaires : ne parler que de la structure, de l'intérêt pour la modélisation & YP \\ \hline
46 & MODEL & arbres : ne parler que de la structure, de l'intérêt pour la modélisation & EB \\ \hline
47 & MODEL & graphes : ne parler que de la structure, de l'intérêt pour la modélisation & EB \\ \hline
48 & MODEL & langages de description : XML, JSON, sérialisation & EB \\ \hline
49 & MODEL & langages de description : XML, JSON, sérialisation & EB \\ \hline
50 & ALGOPROG & de l'algorithme au programme (pgcd), algorithme sur les tableaux, tous les algo de base (min, max, moyenne, recherche, count, etc.),  & YP \\ \hline
51 & ALGOPROG & tris, algorithmes sur les listes & YP \\ \hline
52 & ALGOPROG & algorithmes sur les textes & DJ \\ \hline
53 & ALGOPROG & algorithmes sur les textes & DJ \\ \hline
54 & ALGOPROG & arbres & YP \\ \hline
55 & ALGOPROG & arbres & YP \\ \hline
56 & ALGOPROG & graphes & YP \\ \hline
57 & ALGOPROG & graphes & YP \\ \hline
\end{supertabular}
\end{center}






\end{document}
